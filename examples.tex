\documentclass[11pt]{article}

\usepackage{graphicx}
\usepackage{verbatim}

\usepackage{pctex}

\title{Examples for pctex}
\author{ Martin Trapp }
\date{}

\begin{document}
\maketitle

\noindent
`pctex` provides some useful commands for working with probabilistic circuits. The main purpose of this is reusability and harmonization of notation.

\section{General/Misc}
\begin{itemize}
	\item Log-sum-exp $\lse{i=1}{k}$: \verb!$\lse{i=1}{k}$!
	\item $\poly{N}$: \verb!$\poly{N}$!
	\item Independent RVs $X_1 \indepsym X_2, \indep{X_1}{X_2}$: \verb!$X_1 \indepsym X_2, \indep{X_1}{X_2}$!
	\item Cond. independent RVs $(\cindep{X_1}{X_2}{X_3})$: \verb!$(\cindep{X_1}{X_2}{X_3})$!
\end{itemize}

\section{General graphs}
\begin{itemize}
	\item Graph $\graph$: \verb!$\graph$!
	\item Walk $\walk$: \verb!$\walk$!
	\item Tree $\tree$: \verb!$\tree$!
	\item Vertex set $\vset(\graph)$: \verb!$\vset(\graph)$!
	\item Edge set $\eset(\graph)$: \verb!$\eset(\graph)$!
	\item Node/nodes $\node, \nodes$: \verb!$\node$!
	\item Child/children: $\cnode, \cnodes$: \verb!$\child$!
	\item Children of a node: $\ch{\node}$: \verb!$\ch{\node}$!
	\item Parents of a node: $\pa{\node}$: \verb!$\pa{\node}$!
	\item Neighbours: $\neigh{\node}$: \verb!$\neigh{\node}$!
\end{itemize}

\section{Probabilistic Circuits}
\begin{itemize}
	\item Probabilistic circuit: $\pc$: \verb!$\pc$!
	\item Scope function: $\scope{\node}$: \verb!$\scope{\node}$!
	\item v-tree: $\vtree$: \verb!$\vtree$!
	\item Sum node/nodes: $\snode, \snodes$: \verb!$\snode, \snodes$!
	\item Product node/nodes: $\pnode, \pnodes$: \verb!$\pnode, \pnodes$!
	\item Leaf node/nodes: $\lnode, \lnodes$: \verb!$\lnode, \lnodes$!
	\item Region/regions: $\region, \regions$: \verb!$\region, \regions$!
	\item Partition/partitions: $\partition, \partitions$: \verb!$\partition, \partitions$!
	\item Region-graph: $\rg$: \verb!$\rg$!
\end{itemize}

\section{Tikz / Plotting}
Plotting is based on an adaptation of `tikzlibraryspn.code.tex` by Nicola Di Mauro and Antonio Vergari.

\begin{figure}[h!]
\centering
\begin{tikzpicture}

\sumnode{s1};
\prodnode[below=15pt of s1, xshift=30pt]{p1};
\prodnode[below=15pt of s1, xshift=-30pt]{p2};

\bernode[below=15pt of p1, xshift=-15pt]{v1}{$X_0$};
\bernode[below=15pt of p2, xshift=15pt]{v2}{$\bar{X}_0$};
	
\contnode[below=15pt of p1, xshift=15pt]{v3}{$X_1$};
\contnode[below=15pt of p2, xshift=-15pt]{v4}{$X_1$};
	
\weigedge[right] {s1} {p1} {$\theta_1$};
\weigedge[left] {s1} {p2} {$\theta_2$};

\edge {p1} {v1};
\edge {p2} {v2};
\edge {p1} {v3};
\edge {p2} {v4};
\end{tikzpicture}
\end{figure}

Code for the figure above:
\begin{verbatim}
\begin{tikzpicture}

\sumnode{s1};
\prodnode[below=15pt of s1, xshift=30pt]{p1};
\prodnode[below=15pt of s1, xshift=-30pt]{p2};

\bernode[below=15pt of p1, xshift=-15pt]{v1}{$X_0$};
\bernode[below=15pt of p2, xshift=15pt]{v2}{$\bar{X}_0$};
	
\contnode[below=15pt of p1, xshift=15pt]{v3}{$X_1$};
\contnode[below=15pt of p2, xshift=-15pt]{v4}{$X_1$};
	
\weigedge[right] {s1} {p1} {$\theta_1$};
\weigedge[left] {s1} {p2} {$\theta_2$};

\edge {p1} {v1};
\edge {p2} {v2};
\edge {p1} {v3};
\edge {p2} {v4};  

\end{tikzpicture}

\end{verbatim}


\end{document}
